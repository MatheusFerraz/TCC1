\chapter{Resultados}

Nesta seção, serão apresentados os principais resultados obtidos a partir da realização da pesquisa bibliográfica e também, a partir da revisão sistemática. Inicialmente, serão apresentados os resultados obtidos com a revisão sistemática inerente à temática qualidade dos testes unitários. Posteriormente, serão apresentados insumos bibliográficos para o arcabouço inerente às inspeções de código e à VBSE.

\section{Revisão Sistemática - Qualidade dos Testes Unitários}

A partir da leitura dos arquivos selecionados, foi possível contemplar abordagens quanto:

\begin{itemize}
	\item Ao pensamento que os desenvolvedores devem ter quando se discute implementação de testes unitários.
	\item À existência de práticas e técnicas que devem estar presentes no processo de desenvolvimento de \textit{software} para que os testes unitários sejam efetivos e de qualidade.
	\item Às ferramentas que apoiam a construção de testes unitários.
\end{itemize}

\subsection{Abordagens de pensamento para elaboração de testes unitários}

Primeiramente, deve-se notar que a cultura existente no desenvolvimento de \textit{software} quanto à elaboração de testes unitários deve mudar. Esta é uma atividade fortemente negligenciada na indústria de desenvolvimento. Assim, instituições de prestígio na comunidade de desenvolvimento científico e tecnológico tem apresentado perspectivas importantes a serem consideradas com relação à atividade de implementação de testes unitários. A exemplo disso, tem-se a NASA (\textit{National Aeronautics and Space Administration}), que por desenvolver sistemas críticos, concebeu um pensamento extremamente diferenciado e que atribui a devida importância às atividades de verificação, mais especificamente, a implementação de testes.

É importante destacar que testes unitários são parte integral de um produto e assim, as diferentes versões dos testes unitários também devem ser controladas e gerenciadas como qualquer outra parte do código fonte do produto \cite{nasa}. Nesse sentido, artefatos de teste também são entregáveis. Portanto, os testes unitários bem como os \textit{stubs} e os resultados da execução destes podem ser considerados entregáveis para o cliente como uma maneira de relatar a qualidade presente na construção do produto.

Adicionalmente, as assertivas utilizadas em testes unitários emitem alertas aos desenvolvedores com relação à qualidade do código como um todo, sendo evidenciados aspectos da complexidade ciclomática, número exarcebado de linhas de código em um módulo e invocações de métodos \cite{asserts}.

Considerando as colocações feitas anteriormente, tem-se um alinhamento entre atividades técnicas e a missão de um projeto, o que é fortemente ministrado pela VBSE.

\subsection{Práticas e técnicas para elaboração de testes unitários}

Além das abordagens voltadas para o pensamento dos desenvolvedores citadas anteriormente, é importante evidenciar práticas e técnicas que podem e devem ser empregadas para elaboração de testes unitários mais concisos \cite{nasa}. São elas:

\begin{itemize}
	\item Deve-se criar muitos métodos de teste pequenos ao invés de se criar poucos métodos de teste grandes.
	\item Código de teste deve possuir convenções de nomenclatura.
	\item A ordem de execução dos testes não deve ser fator decisivo.
	\item Testes devem ser auto verificáveis.
	\item A estrutura hierárquica dos testes unitários facilita a compreensão destes.
	\item Deve-se criar estratégias para testar funções mais internas.
	\item \textit{Stubs} devem ser simples e pequenos.
	\item \textit{Stubs} não devem depender de outros \textit{Stubs} que simulam comportamentos de outros módulos.
	\item A arquitetura do \textit{software} deve comportar \textit{stubs}.
	\item A arquitetura do \textit{software} deve abstrair aspectos pertinentes ao \textit{hardware} e ao sistema operacional.
	\item Utilização de ferramentas de análise de cobertura auxiliam a desenvolver novos cenários de teste.
	\item Gráficos e métricas são úteis para analisar a qualidade de testes.
\end{itemize}

É válido ressaltar que as práticas e técnicas apresentadas anteriormente foram derivadas a partir do sucesso contemplado na atividade de implementação de testes unitários promovida pela equipe da NASA.

Outro aspecto do ponto de vista técnico que deve ser verificado em testes unitários é o quesito adequação \cite{adequacao}. Dentro desta temática, é válido ressaltar os quesitos que os testes unitários precisam atender em termos de noção de adequação:

\begin{itemize}
	\item Cobertura de linha, pois espera-se que os métodos de teste unitário elaborados para testar uma determinada unidade exercitem todas as linhas de código da unidade sob teste.
	\item Cobertura de caminho (\textit{path}), pois espera-se que os métodos de teste unitário elaborados exercitem minimamente uma vez cada caminho contemplado na unidade sob teste. Este tipo de cobertura é tratado quando se tem pontos de decisão no código da unidade sob teste.
\end{itemize}

\subsection{Utilização de ferramentas de apoio para elaboração de testes unitários}

Outro aspecto importante na construção e verificação da qualidade dos testes unitários é o uso de ferramentas de apoio \cite{feedback}.

Primeiramente, para que os testes unitários sejam eficazes na identificação de comportamento inadequado de uma determinada unidade, os métodos de teste unitário devem abranger o tanto quanto possível o código do sistema, conforme comentado anteriormente sobre a adequação centralizada nas coberturas de linha e de caminho. Em segundo lugar, os desenvolvedores precisam executar os métodos de teste unitário tão frequentemente quanto possível e assim, a execução deve ser automatizada e rápida.

Existem diversos \textit{frameworks} para implementação de testes unitários para as mais variadas linguagens de programação. A exemplo disso, tem-se o \textit{JUnit} para Java, \textit{CUnit} para linguagem C e o \textit{Rspec} para a linguagem Ruby. Todas estas ferramentas provêem suporte para elaboração de métodos de teste unitário bem como para a execução automatizada da suíte de testes construída.

Adicionalmente, é importante visualizar a cobertura de código \cite{cobertura}. Não basta apenas elaborar uma suíte de testes unitários e executá-la de forma automática, também é necessário avaliar o quanto os métodos de teste unitário estão exercitando o código da unidade sob teste.

A partir do uso de ferramentas de análise de cobertura, é possível identificar mais cenários de teste a serem elaborados e assim, a suíte de testes se torna ainda mais eficaz. Todos os aspectos listados até aqui elevam a qualidade do produto e assim, tem-se a entrega de maior valor para o cliente.

\section{Pesquisa Bibliográfica}

\subsection{Inspeção de Código}

É válido ressaltar que inspeções formais de \textit{software} objetivam a detecção e eliminação de erros em produtos desenvolvidos durante o seu ciclo de desenvolvimento. Nesse sentido, inspeções formais são aplicáveis à qualquer parte do produto de \textit{software}, incluindo requisitos, especificação, \textit{design} e código \cite{inspecao1}.

Segundo \cite{inspecao1}, um \textit{checklist} para inspeção formal de código deve contemplar, dentre outras técnicas, os principais itens:

\begin{itemize}
	\item \textbf{Retorno de métodos}: Retorno de métodos e/ou rotinas muitas vezes pode provocar a interrupção do fluxo do programa, descontinuidade, corrupção de pilha, estouro de memória ou valor incorreto. Assim, verificando este quesito, é possível constatar se todas as rotinas ou métodos retornam valores de maneira correta.

	\item \textbf{Tratamento de interrupção e regiões críticas}: Esta verificação atesta a manutenção de interrupções pelas rotinas correspondentes e por rotinas que dependem dos serviços de interrupção em regiões críticas. É válido ressaltar que este aspecto deve ser profundamente investigado principalmente quando se trata do desenvolvimento de sistemas críticos. Nesse sentido, a inspeção deve localizar todas as rotinas de interrupção de serviços e rotinas que são chamadas por serviços de interrupção.

	\item \textbf{Controle de \textit{loops}}: Este item verifica os loops para garantir que eles possuem fim (exceto quando intencionalmente nunca terminam), evitando ciclos infinitos no programa.

	\item \textbf{Teste de I/O}: Este quesito verifica o I/O de rotinas importantes, especialmente àquelas onde a reentrada deve ser prevenida. Entrada e saída de dados em um programa é um aspecto que deve ser bem avaliado, justamente pelo fato de possuir implicações diretas no uso da CPU (analisando a perspectiva de escalonamento dos processos).

	\item \textbf{Controle de fluxo do programa}: Este item verifica se a sequência do programa está correta. A incorreta utilização de estruturas de controle pode resultar em uma execução inesperada, possibilitando situações de risco no funcionamento do \textit{software}.

	\item \textbf{Código inutilizado}: Esta verificação atesta se não há nenhum código entre marcas de comentário e, portanto, não utilizado (em geral rotinas que serviram para o desenvolvimento e que não são mais utilizadas). Este fator pode ter influência negativa na manutenção futura do código.

	\item \textbf{Variáveis e constantes}: O uso de variáveis e constantes é outro aspecto que deve ser verificado de forma concisa. Deve-se atentar para a correta atribuição dos valores, bem como sua atualização.

	\item \textbf{Comentários de código}: Deve-se verificar se os comentários de fato melhoram a compreensão do código, melhorando a manutenibilidade. Adicionalmente, é importante destacar que os comentários não devem ser ambíguos, podendo resultar em interpretações incorretas.

	\item \textbf{Legibilidade de código}: A legibilidade é fundamental para a manutenção do código. Quanto menor a complexidade na escrita do código, menor o esforço na compreensão.

	\item \textbf{Diretivas ao pré-processador}: Deve-se evitar a utilização indiscriminada das diretivas ao pré-processador no código fonte. Este aspecto evita erros durante a manutenção.

	\item \textbf{Otimização de código}: Deve-se evitar deterinadas otimizações durante a compilação, o que pode gerar um código objeto de maneira inesperada. Em geral, linguagens de programação de alto nível têm aspectos ambíguos que podendo levar a uma dupla interpretação dependendo do nível de otimização selecionado.
\end{itemize}