\chapter{Atividades Futuras}

Este capítulo apresenta o objetivo de medição bem como um breve detalhamento da estratégia a ser utilizada para aplicar o \textit{framework} em organizações. Por fim, tem-se uma breve descrição das organizações onde o \textit{framework} será aplicado durante o TCC 2.

\section{Avaliação da Efetividade do \textit{Framework}}

Para realizar futuramente uma análise da efetividade da aplicação do \textit{framework}, elaborou-se um GQM (\textit{Goal Question Metrics}). A seguir, tem-se uma tabela que resume o objetivo da medição.

\begin{table}[h]
\centering
\begin{tabular}{ | m{8cm} | m{8cm} | } 
\hline
Analisar & A efetividade do \textit{framework} \\ 
\hline
Com o propósito de & Melhorar \\ 
\hline
Com respeito a & Efetividade do \textit{framework} \\ 
\hline
Sob o ponto de vista de & \textit{Stakeholders} do projeto \\ 
\hline
No contexto de & Aumento da qualidade do \textit{software} \\ 
\hline
\end{tabular}
\caption{Tabela Resumo do Objetivo de Medição}\label{table:1}
\end{table}

A partir do quadro resumo que descreve o objetivo de medição, foram elaboradas questões a serem respondidas por meio das medições, bem como as métricas associadas às mesmas. A seguir, tem-se a apresentação das questões e métricas.

\begin{itemize}
	\item \textbf{Questão 1:} Qual a densidade de defeitos identificada a partir da realização da inspeção de código segundo guia do \textit{framework}?
	\subitem Densidade de Defeitos por Unidade do Código

	\item \textbf{Questão 2:} Qual o nível de eficácia dos testes unitários implementados a partir do uso do guia de implementação do \textit{framework}?
	\subitem Percentual de Cobertura de Código

	\item \textbf{Questão 3:} Qual o número de falhas reportados pelos usuários após homologação da \textit{release} entregue?
	\subitem Número de Falhas

	\item \textbf{Questão 4:} Qual o grau de satisfação dos desenvolvedores ao utilizarem os guias de inspeção e de implementação de testes unitários propostos pelo \textit{framework}?
	\subitem Será feito o cômputo das respostas atribuídas pelos desenvolvedores aos itens da pesquisa de satisfação com as seguintes opções de resposta: 1 - Concordo Totalmente; 2 - Concordo Parcialmente; 3 - Discordo Parcialmente e 4 - Discordo Totalmente.

	\item \textbf{Questão 5:} Qual o grau de facilidade percebido quando vai ser feita uma manutenção no código?
	\subitem Índice de Manutenibilidade
\end{itemize}

A coleta de métricas será realizada ao final de cada \textit{sprint}, de maneira a avaliar os resultados e verificar a efetividade do \textit{framework}. Caso ajustes sejam necessários, as modificações serão feitas e novamente, todo o ciclo de atividades será executado na \textit{sprint} seguinte.

Adicionalmente, para o correto uso do \textit{framework}, propõe-se uma etapa a mais no kanbam de implementações da equipe de desenvolvimento, sendo esta a inspeção de código. A funcionalidade só será considerada finalizada, após inspecionada e aprovada por outro integrante da equipe.

\section{Pesquisa de Satisfação dos Desenvolvedores}

Para verificar o nível de satisfação dos desenvolvedores, concebeu-se uma pequena lista com itens a serem julgados pelos desenvolvedores de acordo com as opções disponíveis, citadas como componentes da métrica que responde à questão 4. A seguir, tem-se os itens de pesquisa.

\begin{itemize}
	\item As inspeções de código se mostraram eficazes na identificação de defeitos no código.

	\item Os testes unitários implementados de acordo com o guia, de fato, exercitam o código de maneira eficiente.

	\item As inspeções de código efetuadas segundo o guia e as implementações de testes unitários também feitas de acordo com o guia demonstraram-se onerosos ao processo de desenvolvimento.

	\item O número de falhas percebidos pelo usuário passou a ser menor a cada \textit{release} entregue.

	\item A manutenibilidade do código ficou melhor após as inspeções de código.
\end{itemize}

É importante ressaltar que a pesquisa será aplicada ao final das entregas de \textit{release} e não ao final de cada \textit{sprint}. Dessa maneira, os desenvolvedores poderão assimilar mais experiências para fornecerem respostas mais concisas ao julgar os itens da pesquisa.

\section{Estratégia de Aplicação do \textit{Framework}}

Antes da incorporação das atividades propostas pelo \textit{framework} à rotina das organizações que serão utilizadas como casos, será feita uma reunião com os principais envolvidos de cada projeto. Assim, serão explicitados todos os aspectos do \textit{framework} e a importância dessas atividades para que haja aumento na qualidade do \textit{software}.

Adicionalmente, será construída uma página na wiki ou até mesmo uma simples página web nos ambientes dos projetos que irá dispor a imagem do processo que constitui o \textit{framework}, bem como o detalhamento deste, de forma que os \textit{stakeholders} dos projetos possam efetuar consultas em caso de dúvidas.

Como mencionado no capítulo inerente à metodologia, a pesquisa-ação será o procedimento técnico empregado para analisar o uso do \textit{framework}.

\section{Organizações onde o \textit{Framework} será aplicado}

\subsection{Controladoria Geral do Distrito Federal}

A Controladoria Geral do Distrito Federal (CGDF) teve sua estrutura criada por meio do Decreto nº 36.236, de 1º de janeiro de 2015, pelo Governador Rodrigo Rollemberg. Esta Unidade Gestora possui como missão orientar e controlar a gestão pública, com transparência e participação da sociedade.

Interna à CGDF, há uma coordenação de assuntos tecnológicos, a COTEC, que é responsável por desenvolver e manter sistemas para todas as demais coordenações da CGDF. Além dos sistemas internos, tem-se aplicações que são desenvolvidas por empresas privadas ao governo e entregues à COTEC para serem mantidas.

A equipe da COTEC é pequena, contando com um total de 10 servidores. São 3 servidores na frente de infraestrutura, 3 na frente de administração de banco de dados e apenas 4 na área de desenvolvimento. O \textit{Scrum} e XP foram adotados recentemente na COTEC.

A maior parte dos sistemas desenvolvidos e mantidos pela COTEC são aplicações \textit{Web}, construídas com as linguagens de programação \textit{CSharp} e Java. Adicionalmente, faz-se uso do \textit{framework} \textit{AngularJS} para elaboração da camada de apresentação das aplicações.

Embora já se tenha o mínimo de atividades para controlar o desenvolvimento, a COTEC ainda necessita incorporar uma série de outras práticas. As atividades inerentes à verificação de \textit{software} ainda são incipientes na COTEC, tornando-a um local propício para a aplicação do \textit{framework}.

\subsection{Laboratório Fábrica de \textit{Software} - Campus UnB Gama}

O Laboratório Fábrica de \textit{Software} foi concebido com fins de pesquisa e promoção de excelência no desenvolvimento de \textit{software}. Atualmente, o laboratório já incorpora metodologias ágeis para o desenvolvimento, sendo o \textit{Scrum} e o XP (\textit{Extreme Programming}).

Adicionalmente, o laboratório atua de forma a unir o aprendizado das disciplinas do curso de Engenharia de \textit{Software} do Campus Gama da UnB, e a atuação prática de projetos reais.

O laboratório possui parceria com instituições públicas e privadas, as quais subsidiam pesquisas e projetos utilizando tecnologia \textit{CSharp} e \textit{Android}. Quando necessário, o laboratório também modela processos de negócio utilizando notação BPMN (\textit{Business Process Model and Notation}).

Por se tratar de uma instituição aberta ao âmbito de pesquisa na área de Engenharia de \textit{Software}, o laboratório é um local apropriado para uso e experimentação do \textit{framework}, justamente pelo fato de conciliar interesses práticos do mercado com os fins acadêmicos.